\documentclass[11pt]{article}\usepackage[]{graphicx}\usepackage[dvipsnames]{xcolor}
% maxwidth is the original width if it is less than linewidth
% otherwise use linewidth (to make sure the graphics do not exceed the margin)
\makeatletter
\def\maxwidth{ %
  \ifdim\Gin@nat@width>\linewidth
    \linewidth
  \else
    \Gin@nat@width
  \fi
}
\makeatother

\definecolor{fgcolor}{rgb}{0.345, 0.345, 0.345}
\newcommand{\hlnum}[1]{\textcolor[rgb]{0.686,0.059,0.569}{#1}}%
\newcommand{\hlsng}[1]{\textcolor[rgb]{0.192,0.494,0.8}{#1}}%
\newcommand{\hlcom}[1]{\textcolor[rgb]{0.678,0.584,0.686}{\textit{#1}}}%
\newcommand{\hlopt}[1]{\textcolor[rgb]{0,0,0}{#1}}%
\newcommand{\hldef}[1]{\textcolor[rgb]{0.345,0.345,0.345}{#1}}%
\newcommand{\hlkwa}[1]{\textcolor[rgb]{0.161,0.373,0.58}{\textbf{#1}}}%
\newcommand{\hlkwb}[1]{\textcolor[rgb]{0.69,0.353,0.396}{#1}}%
\newcommand{\hlkwc}[1]{\textcolor[rgb]{0.333,0.667,0.333}{#1}}%
\newcommand{\hlkwd}[1]{\textcolor[rgb]{0.737,0.353,0.396}{\textbf{#1}}}%
\let\hlipl\hlkwb

\usepackage{framed}
\makeatletter
\newenvironment{kframe}{%
 \def\at@end@of@kframe{}%
 \ifinner\ifhmode%
  \def\at@end@of@kframe{\end{minipage}}%
  \begin{minipage}{\columnwidth}%
 \fi\fi%
 \def\FrameCommand##1{\hskip\@totalleftmargin \hskip-\fboxsep
 \colorbox{shadecolor}{##1}\hskip-\fboxsep
     % There is no \\@totalrightmargin, so:
     \hskip-\linewidth \hskip-\@totalleftmargin \hskip\columnwidth}%
 \MakeFramed {\advance\hsize-\width
   \@totalleftmargin\z@ \linewidth\hsize
   \@setminipage}}%
 {\par\unskip\endMakeFramed%
 \at@end@of@kframe}
\makeatother

\definecolor{shadecolor}{rgb}{.97, .97, .97}
\definecolor{messagecolor}{rgb}{0, 0, 0}
\definecolor{warningcolor}{rgb}{1, 0, 1}
\definecolor{errorcolor}{rgb}{1, 0, 0}
\newenvironment{knitrout}{}{} % an empty environment to be redefined in TeX

\usepackage{alltt}
\usepackage[utf8]{inputenc}
\usepackage[T1]{fontenc}
\usepackage{lmodern}
\usepackage{amsmath, amssymb, amsthm, graphicx, physics}
\usepackage{enumitem}
\usepackage{geometry}
\usepackage[dvipsnames]{xcolor}
\definecolor{EvanPink}{rgb}{0.85, 0.13, 0.55}
\usepackage[colorlinks=true,
            linkcolor=OliveGreen,
            urlcolor=EvanPink,
            citecolor=OliveGreen]{hyperref}
\usepackage{cleveref}
\usepackage{etoolbox}
\newenvironment{solution}{%
  \par\noindent\textit{Solution.}\ }{\qed}
\newtheoremstyle{problemstyle}
  {1em}   
  {1em}   
  {}      
  {}      
  {\bfseries} 
  {.}     
  {0.5em} 
  {}      
\theoremstyle{problemstyle}
\newtheorem{problem}{Problem}[section]
\crefname{problem}{Problem}{Problems}
\Crefname{problem}{Problem}{Problems}
\def\bN{\mathbb{N}}
\def\bR{\mathbb{R}}
\def\bZ{\mathbb{Z}}
\def\bQ{\mathbb{Q}}
\def\mM{\mathcal{M}}
\def\mR{\mathcal{R}}
\def\mN{\mathcal{N}}
\def\mE{\mathcal{E}}
\def\mA{\mathcal{A}}
%%%%%%%%%%%%%%%Setting Page Sizes %%%%%%%%%%%%
\setlength{\parindent}{0in}
\setlength{\parskip}{\baselineskip}
\setlength{\textheight}{9in}
\setlength{\textwidth}{6.3in}
\setlength{\topmargin}{-0.6in}
\setlength{\evensidemargin}{-.5in}
\setlength{\oddsidemargin}{0in}
%%%%%%%%%%%%%%%%%%%%%%%%%%%%%%%%%%%%%%%%%%%

%%%%%%%%%%%%To make R commands appear in different Color%%%%%%%%%%
\usepackage{color}
\usepackage{fancyvrb} % Verbatim
\usepackage{Sweave} % for R code
\usepackage{color}
\definecolor{codecolor}{RGB}{114,12,112}
\DefineVerbatimEnvironment{Sinput}{Verbatim}{formatcom=\color{codecolor}}
\DefineVerbatimEnvironment{Soutput}{Verbatim}{formatcom=\color{codecolor}}
\DefineVerbatimEnvironment{Scode}{Verbatim}{formatcom=\color{codecolor}}
\DefineVerbatimEnvironment{Sin}{Verbatim}{formatcom=\color{codecolor}}
\DefineVerbatimEnvironment{Sout}{Verbatim}{formatcom=\color{codecolor}}
%%%%%%%%%%%%%%%%%%%%%%%%%%%%%%%%%%%%%%%%%%%%%%%%%%%%%%%%%%%%%%%%

\IfFileExists{upquote.sty}{\usepackage{upquote}}{}
\begin{document}

{\bf Arkaraj Mukherjee: }
{\bf H.W. 1}

{\bf Solution 1:}
\begin{enumerate}[label=\alph*.]
  \item 
\begin{knitrout}
\definecolor{shadecolor}{rgb}{0.969, 0.969, 0.969}\color{fgcolor}\begin{kframe}
\begin{alltt}
\hldef{instasee} \hlkwb{=} \hlkwd{c}\hldef{(}\hlnum{8}\hldef{,}\hlnum{8}\hldef{,}\hlnum{3}\hldef{,}\hlnum{12}\hldef{,}\hlnum{13}\hldef{,}\hlnum{10}\hldef{,}\hlnum{1}\hldef{,}\hlnum{8}\hldef{,}\hlnum{7}\hldef{)}
\hldef{differences} \hlkwb{=} \hlkwd{diff}\hldef{(instasee)}
\hldef{differences}
\end{alltt}
\begin{verbatim}
## [1]  0 -5  9  1 -3 -9  7 -1
\end{verbatim}
\end{kframe}
\end{knitrout}
  The \texttt{diff} function gives us a vector that has the differences between adjacent terms in the input vector as output. 
  Adding $24$ to every entry in this \texttt{differences} vector gives us the vector \texttt{x} with the number of hours between consecutive logins as we are working with the $24$-hour format. 
\begin{knitrout}
\definecolor{shadecolor}{rgb}{0.969, 0.969, 0.969}\color{fgcolor}\begin{kframe}
\begin{alltt}
\hldef{x} \hlkwb{=} \hldef{differences} \hlopt{+} \hlnum{24}
\hldef{x}
\end{alltt}
\begin{verbatim}
## [1] 24 19 33 25 21 15 31 23
\end{verbatim}
\end{kframe}
\end{knitrout}
  \item 
\begin{knitrout}
\definecolor{shadecolor}{rgb}{0.969, 0.969, 0.969}\color{fgcolor}\begin{kframe}
\begin{alltt}
\hlkwd{max}\hldef{(x)}
\end{alltt}
\begin{verbatim}
## [1] 33
\end{verbatim}
\begin{alltt}
\hlkwd{mean}\hldef{(x)}
\end{alltt}
\begin{verbatim}
## [1] 23.875
\end{verbatim}
\begin{alltt}
\hlkwd{min}\hldef{(x)}
\end{alltt}
\begin{verbatim}
## [1] 15
\end{verbatim}
\end{kframe}
\end{knitrout}
\end{enumerate}

{\bf Solution 2:}
\begin{enumerate}[label=\alph*.]
  \item 
\begin{knitrout}
\definecolor{shadecolor}{rgb}{0.969, 0.969, 0.969}\color{fgcolor}\begin{kframe}
\begin{alltt}
\hldef{scoreSS} \hlkwb{=} \hlkwd{c}\hldef{(}\hlnum{7}\hldef{,}\hlnum{6}\hldef{,}\hlnum{10}\hldef{,}\hlnum{8}\hldef{,}\hlnum{7}\hldef{,}\hlnum{9}\hldef{,}\hlnum{9}\hldef{,}\hlnum{6}\hldef{,}\hlnum{4}\hldef{,}\hlnum{10}\hldef{,}\hlnum{8}\hldef{,}\hlnum{6}\hldef{,}\hlnum{9}\hldef{,}\hlnum{10}\hldef{)}
\hlkwd{max}\hldef{(scoreSS)}
\end{alltt}
\begin{verbatim}
## [1] 10
\end{verbatim}
\begin{alltt}
\hlkwd{mean}\hldef{(scoreSS)}
\end{alltt}
\begin{verbatim}
## [1] 7.785714
\end{verbatim}
\begin{alltt}
\hlkwd{min}\hldef{(scoreSS)}
\end{alltt}
\begin{verbatim}
## [1] 4
\end{verbatim}
\end{kframe}
\end{knitrout}
  \item 
  I can fix this by updating the $4$-th value in the vector \texttt{scoreSS} as follows
\begin{knitrout}
\definecolor{shadecolor}{rgb}{0.969, 0.969, 0.969}\color{fgcolor}\begin{kframe}
\begin{alltt}
\hldef{scoreSS[}\hlnum{4}\hldef{]} \hlkwb{=} \hlnum{5}
\hldef{scoreSS}
\end{alltt}
\begin{verbatim}
##  [1]  7  6 10  5  7  9  9  6  4 10  8  6  9 10
\end{verbatim}
\begin{alltt}
\hlkwd{mean}\hldef{(scoreSS)}
\end{alltt}
\begin{verbatim}
## [1] 7.571429
\end{verbatim}
\end{kframe}
\end{knitrout}
  \item 
\begin{knitrout}
\definecolor{shadecolor}{rgb}{0.969, 0.969, 0.969}\color{fgcolor}\begin{kframe}
\begin{alltt}
\hlkwd{sum}\hldef{(scoreSS} \hlopt{>=} \hlnum{9}\hldef{)}
\end{alltt}
\begin{verbatim}
## [1] 6
\end{verbatim}
\end{kframe}
\end{knitrout}
  \item 
\begin{knitrout}
\definecolor{shadecolor}{rgb}{0.969, 0.969, 0.969}\color{fgcolor}\begin{kframe}
\begin{alltt}
\hldef{scoreSS} \hlopt{>=} \hlnum{9}
\end{alltt}
\begin{verbatim}
##  [1] FALSE FALSE  TRUE FALSE FALSE  TRUE  TRUE FALSE FALSE  TRUE FALSE FALSE
## [13]  TRUE  TRUE
\end{verbatim}
\end{kframe}
\end{knitrout}
  As we can see, \texttt{ScoresSS >= 9} is a vector of \texttt{TRUE} and \texttt{FALSE} values, which seem to be treated as $1$'s and $0$'s respectively when performing arithmetic operations so the sum of all the entries in this vector is just the number of scores which are atleast $9$. 
  We can use this technique to find the number of entries in a vector which satisfy some condition and hence we use it here to find the percentage of scores which are less than $17$
\begin{knitrout}
\definecolor{shadecolor}{rgb}{0.969, 0.969, 0.969}\color{fgcolor}\begin{kframe}
\begin{alltt}
\hlnum{100}\hlopt{*}\hlkwd{sum}\hldef{(scoreSS}\hlopt{<}\hlnum{17}\hldef{)}\hlopt{/}\hlkwd{length}\hldef{(scoreSS)}
\end{alltt}
\begin{verbatim}
## [1] 100
\end{verbatim}
\end{kframe}
\end{knitrout}
  
\end{enumerate}
{\bf{Solution 3}}
\begin{enumerate}[label=\alph*.]
  \item 
\begin{knitrout}
\definecolor{shadecolor}{rgb}{0.969, 0.969, 0.969}\color{fgcolor}\begin{kframe}
\begin{alltt}
\hldef{Shreelakshmibill} \hlkwb{=} \hlkwd{c}\hldef{(}\hlnum{460}\hldef{,}\hlnum{330}\hldef{,}\hlnum{390}\hldef{,}\hlnum{370}\hldef{,}\hlnum{460}\hldef{,}\hlnum{300}\hldef{,}\hlnum{480}\hldef{,}\hlnum{320}\hldef{,}\hlnum{490}\hldef{,}\hlnum{350}\hldef{,}\hlnum{300}\hldef{,}\hlnum{480}\hldef{)}
\hldef{Shreelakshmibill}
\end{alltt}
\begin{verbatim}
##  [1] 460 330 390 370 460 300 480 320 490 350 300 480
\end{verbatim}
\begin{alltt}
\hlkwd{sum}\hldef{(Shreelakshmibill)}
\end{alltt}
\begin{verbatim}
## [1] 4730
\end{verbatim}
\end{kframe}
\end{knitrout}
  \item 
\begin{knitrout}
\definecolor{shadecolor}{rgb}{0.969, 0.969, 0.969}\color{fgcolor}\begin{kframe}
\begin{alltt}
\hlkwd{min}\hldef{(Shreelakshmibill)}
\end{alltt}
\begin{verbatim}
## [1] 300
\end{verbatim}
\begin{alltt}
\hlkwd{max}\hldef{(Shreelakshmibill)}
\end{alltt}
\begin{verbatim}
## [1] 490
\end{verbatim}
\end{kframe}
\end{knitrout}
  These are the minimum and maximum amounts spent in some month respectively.
  \item 
  The number of months with the amount being greater than $400$ can be found as follows:
\begin{knitrout}
\definecolor{shadecolor}{rgb}{0.969, 0.969, 0.969}\color{fgcolor}\begin{kframe}
\begin{alltt}
\hlkwd{length}\hldef{(Shreelakshmibill[Shreelakshmibill}\hlopt{>}\hlnum{400}\hldef{])}
\end{alltt}
\begin{verbatim}
## [1] 5
\end{verbatim}
\end{kframe}
\end{knitrout}
  The percentage too can be found from this like this:
\begin{knitrout}
\definecolor{shadecolor}{rgb}{0.969, 0.969, 0.969}\color{fgcolor}\begin{kframe}
\begin{alltt}
\hlnum{100}\hlopt{*}\hlkwd{length}\hldef{(Shreelakshmibill[Shreelakshmibill}\hlopt{>}\hlnum{400}\hldef{])}\hlopt{/}\hlkwd{length}\hldef{(Shreelakshmibill)}
\end{alltt}
\begin{verbatim}
## [1] 41.66667
\end{verbatim}
\end{kframe}
\end{knitrout}
  \item 
\begin{knitrout}
\definecolor{shadecolor}{rgb}{0.969, 0.969, 0.969}\color{fgcolor}\begin{kframe}
\begin{alltt}
\hldef{freemoney} \hlkwb{=} \hlnum{3000} \hlopt{-} \hldef{Shreelakshmibill}
\hldef{freemoney}
\end{alltt}
\begin{verbatim}
##  [1] 2540 2670 2610 2630 2540 2700 2520 2680 2510 2650 2700 2520
\end{verbatim}
\end{kframe}
\end{knitrout}
  The average money available each month after paying the phone bill is the aveerage of entries in the vector \texttt{freemoney}
\begin{knitrout}
\definecolor{shadecolor}{rgb}{0.969, 0.969, 0.969}\color{fgcolor}\begin{kframe}
\begin{alltt}
\hlkwd{mean}\hldef{(freemoney)}
\end{alltt}
\begin{verbatim}
## [1] 2605.833
\end{verbatim}
\end{kframe}
\end{knitrout}
\end{enumerate}
\end{document}      
