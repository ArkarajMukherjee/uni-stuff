\documentclass[12pt,a4paper]{article}
\usepackage[utf8]{inputenc}
\usepackage[T1]{fontenc}
\usepackage{lmodern}
\usepackage{amsmath, amssymb, amsthm, graphicx}
\usepackage{enumitem}
\usepackage{geometry}
\geometry{margin=1in}
\usepackage[dvipsnames]{xcolor}
\definecolor{EvanPink}{rgb}{0.85, 0.13, 0.55}
\usepackage[colorlinks=true,
            linkcolor=OliveGreen,
            urlcolor=EvanPink,
            citecolor=OliveGreen]{hyperref}
\usepackage{cleveref}
\usepackage{etoolbox}
\newenvironment{solution}{%
  \par\noindent\textit{Solution.}\ }{\qed}
\newtheoremstyle{problemstyle}
  {1em}   
  {1em}   
  {}      
  {}      
  {\bfseries} 
  {.}     
  {0.5em} 
  {}      
\theoremstyle{problemstyle}
\newtheorem{problem}{Problem}[section]
\apptocmd{\problem}{%
  \addcontentsline{toc}{subsection}{Problem~\thesection.\arabic{problem}}%
}{}{}
\crefname{problem}{Problem}{Problems}
\Crefname{problem}{Problem}{Problems}
\setcounter{tocdepth}{2}
\title{\textbf{Analysis I} \\ \large Home Assignments}
\author{Arkaraj Mukherjee \\ B.Math., First Year, ISI Bangalore}
\date{\today}
\def\bN{\mathbb{N}}
\def\bR{\mathbb{R}}
\def\bZ{\mathbb{Z}}
\def\bQ{\mathbb{Q}}
\begin{document}
\maketitle
\newpage
\tableofcontents
\newpage
\section{Home Assignment I (Due: Aug 19, 2025)}
\textit{Here, the set of all natural numbers $\mathbb N$ contains zero as an element. It was confirmed with the professor that "countable" means "either finite or countably infinite."}$\\ \\$
\textbf{Lemma :} A set $A$ is countable if there exists a surjection $f:\mathbb N\to A$.
\begin{proof}
If $A$ is finite then we are done as its countable by definition. Now assume that $A$ is infinite. For each $a\in A$ define $m(a)=\min\{n\in\mathbb{N}:f(n)=a\}$, which exists since $f$ is surjective and the minimum of a nonempty subset of $\mathbb{N}$ exists. Then the function $g:\{m(a):a\in A\}\to A$ with $g(m(a))=a$, is well defined and injective because each $m(a)$ is unique for each $a$ as previously shown and surjective because every $a\in A$ is clearly reached. The domain of $g$ is an infinite subset of $\mathbb{N}$, hence equipotent with $\mathbb{N}$ as shown in class and there exists a bijection between this and $\mathbb N$. Now composing these two gives us a bijection between $\mathbb N$ and $A$ thus proving that $\mathbb N$ and $A$ are equipotent implying that $A$ is countablely infinite hence countable.
\end{proof}
\begin{problem}\label{prob:1-1}
Let $C, D$ be sets with $4$ and $5$ elements respectively. 
Find the number of functions from $C$ to $D$ which are:
(i) injective; (ii) surjective. \\
Similarly, find the number of functions from $D$ to $C$ which are:
(iii) injective; (iv) surjective.
\end{problem}
\begin{solution}
(i) Enumerate the sets as $C=\{c_1,\ldots,c_4\}$ and $D=\{d_1,\ldots,d_5\}$. Now to count the number of injections we can first choose which $4$ elements from $D$ will be in the range of $f$, there are $^5C_4$ ways to do this and as the order matters the number of such injections will be $$4!\times\binom{5}{4}=5!$$
(ii) There are no surjections from $C$ to $D$ as $C$ and $D$ are finite sets where $|C|<|D|$ i.e. $D$ has strictly more values and we can not have all of this in the range of $f$ as the range has atmost as many elements as the domain i.e. $4<5$. (iii) A function from $D$ to $C$ can not be an injection as we have $5$ elements in $D$ but atmost $4$ values their image can be thus we ought to have some element in $C$ thats the image of two distinct elements in $C$. (iv) for it to be surjective we see that all the elements in $C$ must be in its range thus exactly one element in $C$ must be the image of some two distict elements in $D$ and there are $^5C_2$ ways choose these two elements and $4$ ways to choose the element in $C$ which will be their image whereas the $3$ elements that are left in $D$ will be mapped to $3$ distinct elements in $C$ and there are $3!$ ways to do this. Now the multiplicative principle in combinatorics there are $$4\times\binom{5}{2}\times 3!$$surjections.
\end{solution}
\begin{problem}\label{prob:1-2}
Suppose $X$ is a non-empty set and $f : X \to X$ is a function.
Prove or disprove the following:  
(i) $f$ injective $\Leftrightarrow$ $f \circ f$ injective;  
(ii) $f$ surjective $\Leftrightarrow$ $f \circ f$ surjective;  
(iii) $f$ bijective $\Leftrightarrow$ $f \circ f$ bijective.
\end{problem}
\begin{solution}
  (i) This is true. If $f:X\to X$ is injective then so is $f\circ f$ as for $a,b\in X, f(f(a))=f(f(b))\implies f(a)=f(b)\implies a=b$ using $f$'s injectivity twice, this proves the only if part. Now for the if part, when $f\circ f:X\to X$ is injective,for some $a,b\in X,f(a)=f(b)\implies f(f(a))=f(f(b))\implies a=b$ using the fact that $f$ is a function and then the fact that its an injection, thus $f$ is also an injection. (ii) This is true. Using the notation $$f(S):=\{f(x):x\in S\}\text{ for subsets }S\text{ of }X$$we see that $f$ being a surjection is equivalent to the equality $f(X)=X$
 being true. Now for the if part, if $f\circ f$ is a surjection, we see that as $f:X\to X$ we have that $f(X)\subseteq X$ and thus $X=f(f(X))\subseteq f(X)\subseteq X$ giving us $f(X)=X$. For the only if part we can clearly see that $f(X)=X\implies f(f(X))=f(X)=X.$ (iii) This is true as a function is bijecitve iff its injective and its also surjective and we have already seen that $f$ is injective iff $f\circ f$ is and the same goes for surjectivity. \end{solution}
\begin{problem}\label{prob:1-3}
Find three functions $u, v, w$ from $\mathbb{N}$ to $\mathbb{N}$, which are injective and have disjoint ranges.
\end{problem}
\begin{solution}
  Let $u,v,w:\mathbb N\to\mathbb N$ be the functions: $u:k\mapsto 3k,v:k\mapsto 3k+1,w:k\mapsto 3k+2$ these are clearly injective and have disjoint ranges.
\end{solution}
\begin{problem}\label{prob:1-4}
Let $R, S$ be two non-empty sets. Suppose there exists an injective function $g : R \to S$. Show that there exists a surjective function $h : S \to R$.
\end{problem}
\begin{solution}
  As $R$ is nonempty, fix some $x_o\in R$. Now if $g$ is injective then we see that for all $y\in g(R)\subseteq S$, there exists an unique $x\in R$ such that $g(x)=y$ and for these $y$ we set $h(y)=x$ and for $y\in S\setminus g(R)$(if nonempty, otherwise $g$ was a bijection and we can set $h=g^{-1}$ in that case,) we set $h(y)=x_o$. This is well defined as $g$ is an injection and its a surjection as every element in the domain of $g$ is mapped to some element in the range.  
\end{solution}
\begin{problem}\label{prob:1-5}
Suppose $A$ and $B$ are countable sets. Show that $A \cup B$ is countable.
\end{problem}
\begin{solution}
  As $A,B$ are countable there exist surjections $f,g$ from $\mathbb N$ to $A$ and $B$ respectively. Define $h:\mathbb N\to A\cup B$ as $$h(n):=\begin{cases}
    f(k)\text{ if }n=2k+1\text{ for some }k\in\mathbb N\\ g(k)\text{ if }n=2k\text{ for some }k\in\mathbb N
  \end{cases}$$This is clearly a surjection as $A,B\subseteq f(\mathbb N)\cup g(\mathbb N)=h(\mathbb N)$ and thus $A\cup B\subseteq h(\mathbb N)$ and by definititon $A\cup B$ is countable.
\end{solution}
\begin{problem}\label{prob:1-6}
Suppose $A_1, A_2, \dots$ is a sequence of countable sets. Show that
\[
\bigcup_{n=1}^{\infty} A_n = \{ a : a \in A_n \text{ for some } n \in \mathbb{N} \}
\]
is countable. (In other words, a countable union of countable sets is countable.)
\end{problem}
\begin{solution}
  We re index the sets as $A_0,A_1,\ldots$. Now as each $A_n$ is countable, there exists a surjection $f_n:\mathbb N \to A_n$ for all $n \in \mathbb N$. Define $f:\mathbb N \to \bigcup_{n \in \mathbb N} A_n$ by
$$
f(X):=
\begin{cases}
f_{\nu_2(X)}\!\left(\dfrac{\tfrac{X}{2^{\nu_2(X)}}-1}{2}\right) & \text{if } X \neq 0, \\[1ex]
f_0(0) & \text{if } X=0,
\end{cases}
$$
where $\nu_2(X)$ denotes the largest integer $m$ such that $2^m \mid X$. Every $X>0$ can be written uniquely in the form $X=2^m(2k+1)$ with $m,k \in \mathbb N$, and then $f(X)=f_m(k)$. To check surjectivity, let $a \in A_m$ for some $m$. Since $f_m$ is surjective, there exists $k \in \mathbb N$ with $f_m(k)=a$. Setting $X=2^m(2k+1)$ gives $f(X)=a$. Thus $f$ is surjective, and it follows that $\bigcup_{n \in \mathbb N} A_n$ is countable.

\end{solution}
\begin{problem}\label{prob:1-7}
Let $X$ be a non-empty set. Show that the set of all functions from $X$ to $\{0, 1\}$ is in bijective correspondence with the power set of $X$. (Here $X$ need not be a finite set.)
\end{problem}
\begin{solution}
  We define a function $M:\{0,1\}^X\to\mathcal P(X)$ as, $$M(f):=f^{-1}(\{1\})=\{x\in X:f(x)=1\}$$This is injective as $M(f)=M(g)$ implies that for all $x\in X, f(x)=1$ iff $g(x)=1$ and as the only other values these could have is $0$ we see that for all $x\in X$ we also have that $f(x)=0\iff g(x)=0$ thus $f=g$. This is surjective as for any subset $S\subseteq X$ we can find $\{0,1\}^X\ni f:X\to\{0,1\}$ defined as $$f(x):=\begin{cases}
    1\text{ if }x\in S\\ 0\text{ otherwise }
  \end{cases}$$and we see that $M(f)=S.$ As it is both a surjection and an injection we see that its a bijection.
\end{solution}
\begin{problem}\label{prob:1-8}
Let $Y$ be a non-empty set. What is the maximum possible number of distinct sets we can form using $n$-subsets $A_1, A_2, \dots, A_n$ of $Y$, using set theoretic operations of union, intersection, complement in $Y$?  

For instance, when $n = 1$, the answer is $4$: $A_1$, $A_1^c$, $\varnothing = A_1 \cap A_1^c$, $Y = A_1 \cup A_1^c$.  

For $n = 2$, the answer is $16$, where the list goes on something like $A_1$, $A_2$, $A_1 \cap A_2$, $A_1 \cup A_2$, $A_1 \cap A_2^c$, $A_1^c \cap A_2$, $A_1^c \cup A_2^c$, etc.  

Guess the answer for general $n$ and prove it. (Hint: Think of the Venn diagram.)
\end{problem}
\begin{solution}disjoint
  For a set $S$ and a collection of subsets, $X$ we define $\mathfrak{G}(X)$ to be the collection of all sets that are formed with the sets in $X$ via the set theoretic operations of union, intersection and complement in $S$. We claim that the maximum possible number of such sets is $2^{2^n}$ i.e. $|\mathfrak{G}(\{A_1,\ldots,A_n\})|\leq 2^{2^n}$ and a case where this is achieved is for subsets $A_i=\{(x_1,\ldots,x_n):(\forall j\neq i)x_j\in\{0,1\}\text{ and }x_i=1\}$ for all $i=1,\ldots,n$ and $Y=\{(x_1,\ldots,x_n):(\forall i)x_i\in\{0,1\}\}$. For a function $f:\{1,\ldots,n\}\to\{0,1\}$ we define $$\mathfrak{G}(\{A_1,\ldots,A_n\})\ni A(f):=\bigcap_{i=1}^nA_i^{f(i)}\text{ where }A^1:=A\text{ and }A^0:=A^c$$In this case we see that $A(f)=\{(f(1),\ldots,f(n))\}$ are all disjoint sets for different such functions and as there are $2^n$ of these as there are $2^n$ such functions and we can make $2^{2^n}$ distinct sets using these by choosing which ones to include in the union; formally this collection of sets can be written as, $$\left\{\bigcup_{f\in S}A(f)\bigg\vert S\subseteq\{f:\{0,\ldots,n\}\to\{0,1\}\}\right\}$$In this case we also see that this collection is precisely the powerset of $Y$ itself and thus $|\mathfrak{G}(\{A_1,\ldots,A_n\})|=2^{\text{number of subsets of }Y}=2^{2^n}.$ Now we will prove the inequality. Say $Y$ is a set and $A_1,\ldots,A_n$ are subsets, then we claim that, with $A(f)$ defined in the same way, $\mathfrak{G}(\{A_1,\ldots,A_n\})=\mathfrak{G}(\{A(f)\vert f:\{1,\ldots,n\}\to\{0,1\}\})$. This is true because, for all $f,A(f)\in\mathfrak G(\{A_1,\ldots,A_n\})$ and this proves one direction of the set inequality, $\mathfrak{G}(\{A(f)\vert f:\{1,\ldots,n\}\to\{0,1\}\})\subseteq\mathfrak{G}(\{A_1,\ldots,A_n\})$ as anything on the left can have its individual sets $A(f)$ be written in terms of elements on the right with set theoretic operations and clearly $\mathfrak{G}(\text{anything})$ is closed under all the set theoretic operations and thus it contains these sets. We also see that, $$(\forall i)A_i=\bigcup_{\substack{f:\{1,\ldots,n\}\to\{0,1\}\\ f(i)=1}}A(f)\in\mathfrak{G}(\{A(f)\vert f:\{1,\ldots,n\}\to\{0,1\}\})$$and similarly we get the other direction of the set equality. Thus it suffices to show that $|\mathfrak{G}(\{A(f)\vert f:\{1,\ldots,n\}\to\{0,1\}\})|\leq 2^{2^n}$. We first see that the collection in question $\{A(f)\vert f:\{1,\ldots,n\}\to\{0,1\}\}$ is closed under complements in $Y$ and the intersection of any sets in it is empty as all the sets in it are disjoint. Also the complement of the union of some sets in this is again an union of some sets in this as these sets partition $Y$(they are disjoint as for any $f\neq g$ there must exist some $i$ such that $f(i)\neq g(i)$ and we would have that $A(f)\cap A(g)\in A_i^{f(i)}\cap A_i^{g(i)}=A^1\cap A^0=\emptyset$ and also we see that the union of all of these $A(f)$ is $Y$ as any $x\in Y$ is, for all $i,$ either in $A_i$ or $A_i^c$ and we can take the intersection of the ones which contain $x$ to get a elemenent in our collection containing $x$.), thus any expression with unions and intersections and complements in $Y$ can be reduced to a union of some sets in this collection and we have the choices of whether to include some set $A(f)$ in the union and as there are atmost(some of the intersections may be empty) $2^n$ elements in $\{A(f)\vert f:\{1,\ldots,n\}\to\{0,1\}\}$ there can be atmost $2^{2^n}$ such unions giving us the final inequality, $$|\mathfrak{G}(\{A_1,\ldots,A_n\})|=|\mathfrak{G}(\{A(f)\vert f:\{1,\ldots,n\}\to\{0,1\}\})|\leq 2^{2^n}$$
\end{solution}
\begin{problem}\label{prob:1-9}
Let $K = \{0, 1\}$ and $L = \{0, 1, 2, 3\}$. Consider Cartesian products of countably many copies of $K$ and $L$:
\[
M = K \times K \times \cdots, \quad N = L \times L \times \cdots
\]
Show that $M$ and $N$ are equipotent.
\end{problem}
\begin{solution}
  We define a function $f:N\to M$ which maps $(a_0,a_1,\ldots)$ to $(b_0,b_1,\ldots)$ where for all $n\in\mathbb N,$ the representation of the $a_n$ in binary using two digits (a redundant leading zero is allowed )is $b_{2n}b_{2n+1}$ viewing this as a digit and not the product. For example, $$(1,0,2,3,\ldots)\longmapsto(0,1,0,0,1,0,1,1,\ldots)$$where $1$ in binary is written as $01$, $2$ as $10$, $3$ as $11$ and $0$ as $00$. This function is a surjection as given any $b=(b_0,b_1,\ldots)\in M$ we see that $(2b_0+b_1,2b_2+b_3,\ldots)\in N$ is mapped to $b$ as $2^1\cdot b_{2n}+2^0\cdot b_{2n+1}$ is the decimal representation of $b_{2n}b_{2n+1}$(as a number binary following pteviously stated convention), in the decimal system. The function is also injective because each $b\in M$ uniquely determines the $a\in N$ such that $f(a)=b$ as seen above. As it is an injection and also a surjection, it is a bijection and the two sets are equipotent.
\end{solution}
\begin{problem}\label{prob:1-10}
A real number $x$ is said to be a rational number if $x = \frac{p}{q}$, for some integers $p, q$ with $q \neq 0$. Let $\mathbb{Q}$ be the set of rational numbers. Show that $\mathbb{Q}$ is countable.
\end{problem}
\begin{solution}
  The sets $D_q:=\{p/q:p\in\mathbb Z\}$ for $q\in\mathbb N\setminus\{0\}$ are all clearly countable and so is $N\setminus\{0\}$. And as, $$\mathbb Q=\bigcup_{n=1}^\infty D_n$$its countable as its a countable union of countable sets by \hyperref[prob:1-6]{Problem 1.6.}
\end{solution}
\begin{problem}\label{prob:1-11}
Read about “Proof by infinite descent” and write down one such proof.
\end{problem}
We will prove a result due to fermat: The only integer solutions $(x,y,z)$ to the diophantine equation $x^3+2y^3+4z^3=0$ is $(0,0,0).\\$For the sake of contradiction assume that we have some solution $(x,y,z)$ in the integers such that $(x,y,z)\neq (0,0,0)$. Now we see that, $x^3+2y^3+4z^3=0\implies x^2=-2(y^3+2z^3)\implies 2|x^3\implies 2|x$ and we can write $x=2x_*$ for an integer $x_*$. Now substituing this in the equation and dividing by two we get $y^3+2z^3+4x_*^3=0$ and this has the exact same structure as the original equation! So if we have a solution $(x,y,z)$ then we can find another integer solution $(y,z,x_*)=(y,z,x/2)$. We can keep doing this as follows, $$(x,y,z)\to(y,z,x/2)\to(z,x/2,y/2)\to(x/2,y/2,z/2)\to\ldots\to(x/2^n,y/2^n/z/2^n)\to\ldots$$and all of these must be integer solutions by the construction. But this implies that for all $n\in\mathbb N,2^n|x,y,z$ which is a contradiction unless all of $x,y,z$ are zero as a nonzero integer can only have a finite exponent of $2$ in it Thus the only solution is $(0,0,0)$.
\begin{problem}\label{prob:1-12}
Suppose a rabbit moves along a straight line on the lattice points of the plane, making identical jumps every minute (the initial position and the jump vector are unknown). If we can place a trap once every hour at any lattice point of the plane, and the trap captures the rabbit if it is at that point at that moment, can we guarantee capturing the rabbit in a finite amount of time?
\end{problem}

\begin{solution}
Each rabbit path is determined by an initial position $X \in \mathbb{Z}^2$ and a jump vector $Y \in \mathbb{Z}^2$, and can be written as
$$w(X,Y) = \{X + Yt : t \in \mathbb{N} \cup \{0\}\}$$
This gives an injective map from the set of paths to $(\mathbb{Z}^2)^2$, so the set of possible paths is countable. Hence we can enumerate them as $\{w(n) : n \in \mathbb{N}\}$.
We place traps as follows. At stage $1$, place a trap at the point $X+Y \cdot 2^1$ on path $w(1)$. At stage $2$, place traps at $X+Y \cdot 2^2$ for both $w(1)$ and $w(2)$. At stage $3$, place traps at $X+Y \cdot 2^3$ for $w(1), w(2), w(3)$, and so on. In general, at stage $n$ we place traps at $X+Y \cdot 2^n$ for each of $w(1),\dots,w(n)$.
Let $P>0$ denote the number of rabbit jumps that occur in the time it takes us to place one trap. In the problem $P=60$, but the argument works for any positive $P$. By stage $n$, we have placed traps at $2^n$ jumps along each of $w(1),\dots,w(n)$. The total time elapsed is
$$T(n) = P \cdot \frac{n(n+1)}{2}$$
Suppose the rabbit is traveling along path $w(h)$ for some $h \in \mathbb{N}$. By time $T(m)$, it has made at most $Pm(m+1)/2$ jumps. If $m \geq h$, then a trap has been placed at $2^m$ jumps along $w(h)$. If in addition
$$2^m > \frac{Pm(m+1)}{2}$$
then the trap lies ahead of the rabbit on its path and the rabbit will eventually reach it. Since $2^m$ grows exponentially while $\tfrac{Pm(m+1)}{2}$ grows quadratically, this inequality holds for all sufficiently large $m$. Therefore for large enough $m \geq h$, the rabbit is guaranteed to be caught. Thus the rabbit will always be captured in finite time.
\end{solution}
\section{Home Assignment II (Due: Sep 04, 2025)}
\begin{problem}\label{prob:2-1}
  Take $$C=\left\{2-\frac{1}{n}:n\in\bN\right\}\cup\left\{5-\frac{1}{n}:n\in\bN\right\}$$Show that every nonempty subset of $C$ has a minimal element. Determine as to whether the same property holds for $D$ where,$$D=\left\{3-\frac{1}{m}-\frac{1}{n^2}:m,n\in\bN\right\}$$
\end{problem}
\begin{solution}
  Firstly we note that every element in $\{2-1/n:n\in\bN\}$ is less than every element in $\{5-1/n:n\in\bN\}$, so if the subset we choose has nonempty intersection with the first set then it suffices to prove existence of a minimal element for this. Assume that the subset we choose has nonempty intersection with the first set and consider their intersection which is a subset of the first element, it suffices to show that this has a minimal element. Consider the $n\in\bN$ for which $2-1/n$ is in our subset, this being a subset of the natural numbers has a minimal element and we claim that if this is $n_o$ then $2-1/n_o$ is the minimal element we are after, which is clearly true as $2-1/n\leq 2-1/m$ iff $n\leq m$. Now if our chosen subset was disjoint with the first set, we can repeat the same arguement on the second set using $5$ in plcae of $2$. For the second part let $S\subseteq D$ be nonempty. Define
\[
m_0:=\min\{m\in\mathbb N:\exists n\in\mathbb N\text{ with }3-\frac{1}{m}-\frac{1}{n^2}\in S\}\text{ and }
n_0:=\min\{n\in\mathbb N:\;3-\frac{1}{m_0}-\frac{1}{n^2}\in S\}.
\]
If $3-1/m_0-1/n_0^2$ is minimal we are done. Otherwise there exists
$3-1/m-1/n^2\in S$ strictly smaller and by the minimality of $m_0$ we have $m\ge m_0$, hence for such $m$ necessarily $n<n_0$ (otherwise the element would not be strictly smaller). Among
those $m$ pick the least $m_1\geq m_0$ and any corresponding $n_1$, then $n_1<n_0$. If we keep repeating this then it produces a strictly decreasing sequence $n_0>n_1>n_2>\ldots$ of natural numbers which can't go on forever, so we encounter a minimal element eventually. Thus a minimal element exists.

\end{solution}
\begin{problem}\label{prob:2-2}
Find the infimum and supremum of the following subsets of the real line:
\[
A_1 = \Bigl\{3 + \frac{(-1)^n}{n} : n \in \mathbb{N}\Bigr\}, 
\quad 
A_2 = \{x^2 + 1 : 0 \le x \le 1\}.
\]
\end{problem}
\begin{solution}
  First we prove a \textit{lemma} : if a nonempty set $S\subseteq\bR$ has a maximal (or, minimal) element then the supremum (or, infimum) of $S$ is exactly that element. For a proof assume that $\max S=m$ then we see that for all $x\in S$ we have $x\leq m$ thus, $m$ is an upper bound and $\sup S\leq m$. On the other hand we know that $\sup S$ is also an upper bound and as $m=\max S\in S,m\leq\sup S$. This implies that $m=\sup S.$ For the infimum and minimal element its very similar. Now we see that in $A_1$, a maximal element exists which is $3.5=3+(-1)^2/2$ as $3+(-1)^1/1=2<3.5$ and for all $n>2,$ $$3+\frac{(-1)^n}{n}\leq 3+\frac{1}{n}< 3+\frac{1}{2}$$Thus the supremum of $A_1$ is $3.5$. We similarly see that the minimal element of $A_1$ is $3+(-1)^1/1=2$ as for all $n>1,$ $$3+\frac{(-1)^1}{1}<3+\frac{(-1)}{n}\leq 3+\frac{(-1)^n}{n}$$and thus this must be the infimum.  It was shown in class that for $a,b\geq 0$ the inequalities $a\geq b$ and $a^2\geq b^2$ are equivalent, we can add $1$ to both sides of the second inequality to get $a\geq b\iff a^2+1\geq b^2+1$ for all $a,b\geq 0$ and thus we clearly see that for all $x$ such that $0\leq x\leq 1$ we have, $1=0^2+1\leq x^2+1\leq 1^2+1=2$ and thus $1$ and $2$ are the minimal and maximal elements of $A_2$ respectively and as shown previously they must also be the infimum and supremum respectively.
\end{solution}
\begin{problem}\label{prob:2-3}
Let \(A,B\) be non-empty, bounded subsets of \(\mathbb{R}\). Define
\[
A+B = \{a+b : a \in A, b \in B\}, \quad 
A-B = \{a-b : a \in A, b \in B\}, \quad 
AB = \{ab : a \in A, b \in B\}.
\]
Show that these sets are bounded. Determine which of the following statements are true and which are false in general (prove your claim):

(a) \(\sup(A \cup B) = \max\{\sup A, \sup B\}\).  
(b) \(\sup(A \cap B) = \min\{\sup A, \sup B\}\).  
(c) \(\sup(A+B) = \sup A + \sup B\).  
(d) \(\sup(A-B) = \sup A - \sup B\).  
(e) \(\sup(AB) = (\sup A)(\sup B)\).  
\end{problem}
\begin{solution}
  Since $A,B$ are bounded pick $M,N>0$ with $|a|\le M$ for all $a\in A$ and $|b|\le N$ for all $b\in B$. Then for all $a\in A,b\in B$ we have $|a+b|\le M+N$, $|a-b|\le M+N$, and $|ab|\le MN$. Hence $A+B,A-B,AB$ are bounded. The proofs for $A\cup B,A\cap B$ being bounded are shown below.
\newline
  (a) The claim is true. For any $x\in A\cup B$ we see that $x$ is either in $A$ or in $B$ and if $x\in A$ then $|x|\leq M\leq\max\{M,N\}$ and if $x\in B$ then $|x|\leq N\leq\max\{M,N\}$ thus for all $x\in A\cup B$ we have that $|x|\leq\max\{M,N\}$ so these are bounded and a supremum exists. Let $m=\sup(A\cup B)$, then for all $x\in A,B$ we have that $x\leq m$ thus $m\geq\sup A,\sup B$ which implies that $m\geq \max\{\sup A,\sup B\}.$ Now if $x\in A\cup B$ then $x\in A$ or $x\in B$ and if $x\in A$ then $x\leq \sup A\leq m$ and if $x\in B$ then $x\leq\sup B\leq m$ thus we see that $\sup(A\cup B)\leq m$. This implies that $\sup(A\cup B)=\max\{\sup A,\sup B\}$. \newline 
  (b) The claim is false. For any $x\in A\cap B$ we see that $x\in A$ so $|x|\leq M$ and thus its bounded. Now consider $A=\{0,2\}$ and $B=\{0,1\}$, then we have that $\min\{\sup A,\sup B\}=\min\{1,2\}=1\neq 0=\sup\{0\}=\sup A\cap B$. \newline
  (c) This claim is true. Say $a=\sup A$ and $b=\sup B$ for brevity. Then for all $x+y\in A+B$ with $x\in A,y\in B$ we see that $x\leq a$ and $y\leq b$ which implies that $x+y\leq a+b$ thus $a+b$ is an upper bound for $A+B.$ Now given any $\varepsilon>0$ we can find $x_o+y_o\in[a+b-\varepsilon,a+b]$ by choosing $x_o$ in $A$ and $y_o$ in $B$ such that $x_o\in[a-\varepsilon/2,a]$ and $y_o\in[b-\varepsilon/2,b]$ (which exist as we defined $a,b$ to be the supremums of $A,B$ respectively) as $$a+b-\varepsilon=(a-\varepsilon/2)+(b-\varepsilon/2)\leq x_o+y_o\leq a+b$$This is enough to conclude that $a+b$ is the supremum of $A+B$. (It was shown in class that if for some nonempty set $S\subseteq\bR,s$ is an upper bound such that for all $\varepsilon>0$ we have $S\cap[s-\varepsilon,s]\neq\emptyset$ then its the supremum.) \newline
  (d) This is false. Consider $A=\{0\}$ and $B=[-1,1]$. Then we have that $A-B=\{0-x:x\in[-1,1]\}=[-1,1]$ and thus $\sup(A-B)=1$. But as $\sup A=0$ and $\sup B=1$, $\sup(A-B)=1$ is not equal to $\sup A-\sup B=0-1=-1$ disproving the statement. \newline
  (e) This is false. Consider $A=\{-1\}$ and $B=[-1,1]$. We see that $AB=\{-x:x\in[-1,1]\}=[-1,1]$ and thus $\sup AB=1$ but this is not equal to $(\sup A)(\sup B)=(-1)(1)=-1$ disproving the statement.
\end{solution}
\begin{problem}\label{prob:2-4}
Let \(\{t_n\}_{n \ge 1}\) be a sequence defined by
\[
t_1 = 2, 
\quad 
t_{n+1} = \tfrac{1}{2}\Bigl(t_n + \frac{2}{t_n}\Bigr) \quad (n \ge 1).
\]
Show that \(\{t_n\}\) is a convergent sequence, converging to \(\sqrt{2}\).
\end{problem}
\begin{solution}
  Firstly by AM-GM inequality we see that for all $n\geq 1$, $$t_{n+1}=\frac{t_n+2/t_n}{2}\geq\sqrt{t_n\cdot\frac{2}{t_n}}=\sqrt{2}$$and $t_1=2\geq\sqrt{2}$ as well so the sequence is bounded below by $\sqrt{2}$. From $t_n\geq \sqrt{2}$ we have, $t_n^2\geq 2$ which implies that $t_n\geq 2/t_n$. Using this, for all $n\geq 1$ we have, $$t_{n+1}=\frac{t_n+2/t_n}{2}\leq\frac{t_n+t_n}{2}=t_n$$i.e. $\{t_n\}_{n\geq 1}$ is a decreasing sequence. As this is also bounded below, by a theorem proved in class (all decreasing sequences that are bounded above are convergent) this must converge to some limit $L$ and we must also have that $L\neq 0$ as $L\geq\sqrt{2}$ by another theorem proved in class (if $\{x_n\}_{n\geq 1}$ is bounded below by $m$ and it converges, then $\lim_{n\to\infty}x_n\geq m$ as well.) Taking the limit on both sides and using the algebra of limits of sequences as shown in class we have, $$L=\lim_{n\to\infty}t_n=\lim_{n\to\infty}t_{n+1}=\lim_{n\to\infty}\frac{1}{2}\left(t_n+\frac{2}{t_n}\right)=\frac{1}{2}\left(\lim_{n\to\infty}t_n+\frac{2}{\lim_{n\to\infty}t_n}\right)=\frac{1}{2}\left(L+\frac{2}{L}\right)$$which is a quadratic in $L$ with solutions $\pm\sqrt{2}$ and as $L\geq\sqrt{2}$ it must be $\sqrt{2}.$ We have thus shown that it converges to $\sqrt{2}$.
\end{solution}
\begin{problem}\label{prob:2-5}
Show that 
\[
\lim_{n \to \infty}\Bigl(1 + \tfrac{1}{n}\Bigr)^n
\]
exists. (Hint: Prove that it is a monotone bounded sequence.)
\end{problem}
\begin{solution}
  For all $n\geq 1$, by the binomial theorem we have that, $$\left(1+\frac{1}{n}\right)^n=\sum_{k=0}^n\binom{n}{k}\frac{1}{n^k}=\sum_{k=0}^n\frac{n(n-1)\ldots(n-k+1)}{k!}\frac{1}{n^k}\leq\sum_{k=0}^n\frac{n^k}{k!}\cdot\frac{1}{n^k}\leq\sum_{k=0}^n\frac{1}{k!}$$ $$=1+\frac{1}{1}+\frac{1}{2}+\frac{1}{2\cdot 3}+\frac{1}{2\cdot 3\cdot 4}+\ldots\leq 1+\left(\frac{1}{1}+\frac{1}{2}+\frac{1}{2\cdot 2}+\frac{1}{2\cdot 2\cdot 2}+\ldots\right)=1+2=3$$Thus the sequence is bounded above by $3.$ Using the AM-GM inequality we see that, $$\frac{1+\underbrace{\left(1+\frac{1}{n}\right)+\ldots+\left(1+\frac{1}{n}\right)}_{\text{n times}}}{n+1}\geq\left(1\cdot\underbrace{\left(1+\frac{1}{n}\right)\cdot\ldots\cdot\left(1+\frac{1}{n}\right)}_{\text{n times}}\right)^{\frac{1}{n+1}}$$simplifying this we see that its just the inequality, $$1+\frac{1}{n+1}\geq \left(1+\frac{1}{n}\right)^{\frac{n}{n+1}}$$and now raising both sides to the exponent of $n+1$ this reads as $$\left(1+\frac{1}{n+1}\right)^{n+1}\geq\left(1+\frac{1}{n}\right)^n$$so the sequence is increasing. Now as shown in class we can conclude that its convergent from the fact that its increasing and bounded above.
\end{solution}
\begin{problem}\label{prob:2-6}
Show that there exists a unique positive real number \(x\) such that \(x^3 = 2\).
\end{problem}
\begin{solution}
Let $S:=\{x\in\bR:x^3<2\}$. The set $S$ is nonempty because $1^3=1<2$ so $1\in S$. It is bounded above because if $x\geq 2$ then $x^3\geq 8>2$, hence $x\notin S$, so every $x\in S$ satisfies $x<2$ and thus $2$ is an upper bound. By the completeness axiom, $s:=\sup S$ exists. We claim $s^3=2$.

Suppose first that $s^3<2$. For any $\varepsilon>0$ with $\varepsilon\leq 1$ we expand
\[
(s+\varepsilon)^3 = s^3 + 3s^2\varepsilon + 3s\varepsilon^2 + \varepsilon^3 \leq s^3 + (3s^2+3s+1)\varepsilon.
\]
Since $s^3<2$, choose $\varepsilon>0$ so small that $(3s^2+3s+1)\varepsilon < 2-s^3$. Then $(s+\varepsilon)^3 < 2$, so $s+\varepsilon \in S$. But $s+\varepsilon > s$, contradicting the definition of $s$ as the least upper bound of $S$. Therefore $s^3\geq 2$.

Suppose instead that $s^3>2$. For any $\varepsilon>0$ with $\varepsilon\leq 1$ we expand
\[
(s-\varepsilon)^3 = s^3 - 3s^2\varepsilon + 3s\varepsilon^2 - \varepsilon^3 \geq s^3 - (3s^2+1)\varepsilon.
\]
Since $s^3>2$, choose $\varepsilon>0$ so small that $(3s^2+1)\varepsilon < s^3-2$. Then $(s-\varepsilon)^3>2$ and as $s$ is the supremum there exists some $x\in S$ such that $s-\varepsilon\leq x\leq s$ which implies that $2<(s-\varepsilon)^3\leq x^3<2\implies 2<2$, a contradiction, thus $s^3\leq 2$. 
Combining both inequalities we conclude $s^3=2$, so such a real number exists. For uniqueness, suppose $a^3=b^3=2$ with $a,b>0$. Then $(a-b)(a^2+ab+b^2)=a^3-b^3=0$. Since $a^2+ab+b^2>0$, we must have $a-b=0$, so $a=b$. Hence there exists exactly one positive real $s$ with $s^3=2$.
\end{solution}
\begin{problem}\label{prob:2-7}
Prove that the following sequences are convergent:
\[
a_n = \prod_{k=1}^n \Bigl(1 - \frac{1}{k+1}\Bigr), \quad n \in \mathbb{N};
\]
\[
b_n = \frac{1}{n+1} + \frac{1}{n+2} + \cdots + \frac{1}{n+n}, \quad n \in \mathbb{N}.
\]
\end{problem}
\begin{solution}
  For all $n\in\bN$, $$a_n=\prod_{k=1}^n\left(1-\frac{1}{k+1}\right)=\prod_{k=1}^n\frac{k}{k+1}=\frac{1}{2}\cdot\frac{2}{3}\cdot\ldots\cdot\frac{n}{n+1}=\frac{1}{n+1}$$Thus $a_n=1/(n+1)$ and it converges to $\lim_{n\to\infty}a_n=\lim_{n\to\infty}1/(n+1)=0$ so its convergent. We will show that $b_n$ is bounded above and increasing, this is sufficient as we can conclude convergence from this as discussed in class. For boundedness, $$b_n=\sum_{k=1}^n\frac{1}{n+k}\leq\sum_{k=1}^n\frac{1}{n+1}=\frac{n}{n+1}<1$$so its bounded above by $1$. To prove that its increasing its sufficnent to show that $b_{n+1}-b_n>0$ for all $n\in\bN$, if we write these out we have, $$b_{n+1}-b_n=\left(\frac{1}{n+2}+\frac{1}{n+3}\ldots+\frac{1}{2n+2}\right)-\left(\frac{1}{n+1}+\frac{1}{n+2}+\ldots+\frac{1}{2n}\right)$$ $$=\frac{1}{2n+2}+\frac{1}{2n+1}-\frac{1}{n+1}>\frac{1}{2n+2}+\frac{1}{2n+2}-\frac{1}{n+1}=0$$so we are done.
\end{solution}
\begin{problem}\label{prob:2-8}
Prove that the sequence
\[
c_n = 5 + (-1)^n\Bigl(2 + \frac{1}{n}\Bigr)
\]
is not convergent.
\end{problem}
\begin{solution}
  Consider the subsequences $\{c_{2k}\}_{k\geq 1}$ and $\{c_{2k-1}\}_{k\geq 1}$. We have, $$c_{2k-1}=5+(-1)^{2k-1}(1+1/(2k-1))=5-1-1/(2k-1)=4-1/(2k-1)$$taking the limits,$$\lim_{k\to\infty} c_{2k-1}=\lim_{k\to\infty}(4-1/(2k-1))=\lim_{k\to\infty}4+(-1)\cdot\lim_{k\to\infty}1/(2k-1)=4-0=4$$and similarly for the other subsequence we have $$c_{2k}=5+(-1)^{2k}(1+1/2k)=5+1+1/2k=6+1/2k$$again taking the limits $$\lim_{k\to\infty}c_{2k}=\lim_{k\to\infty}(6+1/2k)=\lim_{k\to\infty}6+\lim_{k\to\infty}1/2k=6+0=6$$So we have two subsequences with different limits which implies that the sequence can't converge as then they'd have the same limits by the theorem: all subsequences of a converging sequence also converge to the same limit as that sequence (this was shown in class).
\end{solution}
\begin{problem}\label{prob:2-9}
Suppose \(\{x_n\}\) is a real sequence. For \(n \ge 1\) define the averages
\[
y_n = \frac{1}{n}\sum_{k=1}^n x_k.
\]
Show that if \(\{x_n\}\) converges, then \(\{y_n\}\) also converges. However, the converse is not true.
\end{problem}
\begin{solution}
  Assume that $x_n$ converges to $x$. Then for any $\varepsilon>0$ there exists $N\in\bN$ such that for all $n>N$ we have $|x-x_n|<\varepsilon$. Now for $M>N$, using the triangle inequality we have, $$\left|x-\frac{1}{M}\sum_{k=1}^Mx_k\right|=\frac{1}{M}\left|\sum_{k=1}^M(x-x_k)\right|\leq\frac{1}{M}\left|\sum_{k=1}^{N}(x-x_k)\right|+\frac{1}{M}\sum_{k=N+1}^M|x-x_k|$$ $$\leq\frac{1}{M}\left|\sum_{k=1}^N(x-x_k)\right|+\frac{(M-N)\varepsilon}{M}\leq\frac{1}{M}\left|\sum_{k=1}^N(x-x_k)\right|+\varepsilon$$As the quantity $G=|\Sigma_{k=1}^N(x-x_k)|$ is fixed we can find $M'\in\bN$ such that $G/M'<\varepsilon$ and clearly for all $K>M'$ we have that $G/K\leq G/M'<\varepsilon$. Thus for all $K>M'$ we have, $$\left|x-\sum_{k=1}^Kx_k\right|\leq\frac{G}{K}+\varepsilon\leq\frac{G}{M'}+\varepsilon<\varepsilon+\varepsilon=2\varepsilon$$which can be made arbitrarily small and thus we see that for all $\varepsilon>0$ there exists $M'\in\bN$ such that for all $K>M'$ we have $|y_n-x|<\varepsilon$ so by the definition of convergence, $y_n$ also converges to $x$. We provide a counterexample for the converse i.e. a sequence $\{x_n\}$ that does not converge but for which $\{y_n\}$ does. Such an example is the sequence $\{x_n\}$ with $x_{2n}:=1,x_{2n-1}:=0$ for $n\in\bN$, which clearly does not converge but the average,$$y_n=\frac{1}{n}\sum_{k=1}^nx_k=\frac{\lfloor n/2\rfloor}{n}$$does converge which we can conclude from the fact that its bounded on both sides by convergent sequences that converge to the same limit as follows, $$\text{(has limit half as well clearly) }1/2-1/n=\frac{\frac{n}{2}-1}{n}\leq y_n\leq\frac{n/2}{n}=1/2\text{ (constant sequence)}$$where the inequalities are true as for any real $x,$ the integer part $\lfloor x\rfloor\in(x-1,1].$ As we provided a counterexample to the converse it can't be true.
\end{solution}
\begin{problem}\label{prob:2-10}
Find all functions \(f : \mathbb{R} \to \mathbb{R}\) satisfying
\[
f(x+y) = f(x) + f(y), 
\quad 
f(xy) = f(x)f(y)
\]
for all \(x,y \in \mathbb{R}\). (Hint: You may need order properties and completeness of \(\mathbb{R}\).)
\end{problem}
\begin{solution}
  \textit{The only solutions are the identity function and the identically zero function. The proof works solely because of completeness axiom and the order defined on the reals which directly or indirectly implies the existence of roots down the line which is crucial to this proof.}\newline
  Firstly we see that $f(1)=f(1\cdot 1)=f(1)^2\implies f(1)=0$ or, $1$ and if its $0$ then everything is i.e. for all $x$ we would have $f(x)=f(x\cdot 1)=f(x)\cdot f(1)=f(x)\cdot 0=0$ so $f$ would be identically zero. We work with the case where $f(1)=1$ now. From the definition, $f(0)=f(0+0)=f(0)+f(0)\implies f(0)=0$ and this is the only element that is mapped to zero as otherwise if we had some nonzero $v$ being mapped to zero, this would imply that, for all $f(1)=f(v\cdot(1/v))\cdot f(v)=f(1/v)\cdot 0=0$ which is not possible as we assumed $f(1)$ to be $1$. We also see that $f(-x)=f((-1)\cdot x)=f(-1)\cdot f(x)$ and as $f(-1)+f(1)=f(-1+1)=f(0)=0\implies f(-1)=-1$ we see that $f(-x)=-f(x)$, this shows us that $f$ is uniquely determined by its values on the positive reals. Now for positive integers $n$, $$f(n)=f\left(\underbrace{1+\ldots+1}_{\text{n times}}\right)=\underbrace{f(1)+\ldots+f(1)}_{\text{n times}}=\underbrace{1+\ldots+1}_{\text{n times}}=n$$and we can generalise this to negative integers too using the fact that $f(-n)=-f(n)=-n$. As $f(0)=0$ too we see that $f$ is the identity on the integers. Now for nonzero integers $n$ and any integer $m$ we see that, $$f(m/n)\cdot n=f(m/n)\cdot f(n)=f(m)=m\implies f(m/n)=m/n$$thus $f$ acts as the identity function on the rationals as well. Now for positive $x$ we can see that $f(x)=f(\sqrt{x}\cdot\sqrt{x})=f(\sqrt{x})^2>0$ (the existence of these $\sqrt{x}$ can be proven just like in \hyperref[prob:2-6]{Problem 2.6} or, by using the intermediate value theorem on the continous function $x\mapsto x^2$ which was all discussed in class. The strict inequality is due to $\sqrt{x}$ being nonzero for nonzero $x$.) We will now show that $f(x)=x$ for all $x\in\bR$. For the sake of contradiction assume that $f(k)<k$ for some $k\in\bR$ i.e. $f(k)=k-\varepsilon$ for some $\varepsilon>0$. Now for any rationals $r<k$ i.e. $k-r>0$ we must have that $0<f(k-r)=f(k)-f(r)=f(k)-r=k-r-\varepsilon\implies \varepsilon<k-r$ which can't be true as we can choose a rational $r$ in $(k-\varepsilon/2,k)$ (we have shown in class that any open interval in the reals contains atleast one rational) for which we would have $k-r<\varepsilon/2$ but this implies that $\varepsilon<\varepsilon/2$ which is false for positive $\varepsilon$ and we have a contradiction so $f(k)\geq k$ for all reals $k$. Now again, with the goal of showing that $f(k)\leq k$ for all reals $k$, for the sake of contradiction assume that for some real $k$ we have $f(k)>k$ i.e. $f(k)=k+\varepsilon$ for some $\varepsilon>0.$ This time consider rationals $r>k$ i.e. $r-k>0$. For these we have that, $0<f(r-k)=f(r)-f(k)=r-f(k)=r-k-\varepsilon\implies r-k>\varepsilon$ which again can't be true as we can similarly choose some rational $r\in(k,k+\varepsilon/2)$ for which we would have $r-k<\varepsilon/2$ implying that $\varepsilon/2<\varepsilon$ which is false for positive $\varepsilon$ and thus we have a contradiction so $f(k)\leq k$ for all reals $k$. As we showed that $k\leq f(k)\leq k$ for all reals $k$ we have proved that $f(k)=k$ for all reals $k$ and we are done. So the functions $f$ satisfying the conditions of the problem are exactly the identity function i.e $x\mapsto x$ and the identically zero function i.e. $x\mapsto 0$.
\end{solution}
\end{document}
