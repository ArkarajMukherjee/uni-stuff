\documentclass[12pt,a4paper]{article}
\usepackage[utf8]{inputenc}
\usepackage[T1]{fontenc}
\usepackage{lmodern}
\usepackage{amsmath, amssymb, amsthm, graphicx}
\usepackage{enumitem}
\usepackage{geometry}
\geometry{margin=1in}
\usepackage[dvipsnames]{xcolor}
\definecolor{EvanPink}{rgb}{0.85, 0.13, 0.55}
\usepackage[colorlinks=true,
            linkcolor=OliveGreen,
            urlcolor=EvanPink,
            citecolor=OliveGreen]{hyperref}
\usepackage{cleveref}
\usepackage{etoolbox}
\newenvironment{solution}{%
  \par\noindent\textit{Solution.}\ }{\qed}
\newtheoremstyle{problemstyle}
  {1em}   
  {1em}   
  {}      
  {}      
  {\bfseries} 
  {.}     
  {0.5em} 
  {}      
\theoremstyle{problemstyle}
\newtheorem{problem}{Problem}[section]
\apptocmd{\problem}{%
  \addcontentsline{toc}{subsection}{Problem~\thesection.\arabic{problem}}%
}{}{}
\crefname{problem}{Problem}{Problems}
\Crefname{problem}{Problem}{Problems}
\setcounter{tocdepth}{2}
\title{\textbf{Analysis I} \\ \large Home Assignment V}
\author{Arkaraj Mukherjee \\ B.Math., First Year, ISI Bangalore}
\date{\today}
\def\bN{\mathbb{N}}
\def\bR{\mathbb{R}}
\def\bZ{\mathbb{Z}}
\def\bQ{\mathbb{Q}}
\begin{document}
\maketitle
\newpage
\tableofcontents
\newpage
\section{Home Assignment V (Due: Dec 4, 2025)}
\begin{problem}\label{prob:5-1}
(Discrete L'Hospital): Let $\{a_n\}_{n\geq1}, \{b_n\}_{n\geq1}$ be sequences of non-zero real numbers converging to 0. Suppose $b_n > b_{n+1}$ for all $n$ and $v := \lim_{n\to\infty} \frac{a_{n+1}-a_n}{b_{n+1}-b_n}$ exists as a real number. Show that $\lim_{n\to\infty} \frac{a_n}{b_n}$ exists and equals $v$. Give one such example.
\end{problem}
\begin{solution}
 Given any $\varepsilon>0$ there exists $N$ such that $\forall n>N,$
	$$v-\varepsilon<\frac{a_{n+1}-a_n}{b_{n+1}-b_n}<v+\varepsilon\implies(v-\varepsilon)(b_{n+1}-b_n)>a_{n+1}-a_n>(v+\varepsilon)(b_{n+1}-b_n)$$
Where the signs flipped because $b_{n+1}-b_n<0$ as stated in the problem.
We sum both sides of this inequality for $n,n+1,\ldots,m$ to get,
	$$(v-\varepsilon)\sum_{k=n}^m(b_{k+1}-b_k)>\sum_{k=n}^m(a_{k+1}-a_k)>(v+\varepsilon)\sum_{k=n}^m(b_{k+1}-b_k)$$
	$$\implies (v-\varepsilon)(b_{m+1}-b_n)>a_{m+1}-a_n>(v+\varepsilon)(b_{m+1}-b_n)$$
Thus this holds for all $m\geq n$ and taking the limits as $m\to\infty$ on all of the terms in this inequality we get, 
	$$(v-\varepsilon)\geq (-b_n)\geq -a_n\geq (v+\varepsilon)(-b_n)\implies v-\varepsilon\leq\frac{a_n}{b_n}\leq v+\varepsilon$$
	which is enough to conclude that $a_n/b_n\longrightarrow v$ as well.
\end{solution}
\begin{problem}\label{prob:5-2}
Suppose $k \in \mathbb{N}$ and $b_1, b_2, \dots, b_k$ are strictly positive real numbers. Show that
(i) $\lim_{n\to\infty} b_j^{\frac{1}{n}} = 1$. (ii) $\lim_{n\to\infty}(b_1^n+b_2^n+\dots+b_k^n)^{\frac{1}{n}} = b$ where $b = \max \{b_j : 1 \leq j \leq k\}$.
\end{problem}
\begin{solution}
	For (i), let $x>0$ and consider the two cases : $x\geq 1$ and $x<1$. For the first case, by bernoulli's inequality we see that,
	$$x=(x^{1/n})^n=(1+(x^{1/n}-1))^n\geq 1+n(x^{1/n}-1)\implies x^{1/n}-1\leq\frac{x-1}{n}$$
	Here we have used the fact that $x^{p}\geq1$ for any positive $p$, making $x^{1/n}-1$ nonnegative. Thus we can use the squeeze theorem on the following inequality to conclude,
	$$0\leq x^{1/n}-1\leq \frac{x-1}{n}$$
For the second case, we use another one of  bernoulli's inequality again to get
	$$x=(1-(1-x^{1/n}))^n\geq 1-n(1-x^{1/n})\implies 1-x^{1/n}\leq\frac{1-x}{n}$$
	where we used the fact that $0<x^p<1$ for any positive $p$, making $1-x^{1/n}$ positive. Again we use the squeeze theorem on the following expression to conclud,e
	$$0\leq 1-x^{1/n}\leq \frac{1-x}{n}$$
	To prove (ii) we will also use squeeze theorem. As $b$ is the maximum of these $b_i$ we have,
	$$b^n<b_1^n+\ldots+b_k^n\leq b^n+\ldots+b^n=nb^n$$
	$$\implies b<(b_1^n+\ldots+b_k^n)^{1/n}\leq n^{1/n}b$$
Using the fact $n^{1/n}\longrightarrow 1$, which is proven in the immediate next exercise, we can conclude using squeeze theorem.
\end{solution}
\begin{problem}\label{prob:5-3}
Show that $\lim_{n\to\infty} n^{\frac{1}{n}} = 1$.
\end{problem}
\begin{solution}
	For large enough($n\geq 3)$ we can say that $n^{1/n}\geq 1$ and then using the binomial theorem
	$$n=(1+(n^{1/n}-1))^n=\sum_{k=0}^n\binom{n}{k}(n^{1/n}-1)^k\geq\frac{n(n-1)}{2}(n^{1/n}-1)^2$$
	$$\implies 1\leq n^{1/n}\leq 1+\sqrt{\frac{2}{n-1}}$$
And yet again we conclude using the squeeze theorem.
\end{solution}
\begin{problem}\label{prob:5-4}
Let $f : \mathbb{R} \to \mathbb{R}$, be a continuous function satisfying
$f(3x) = f(x), \forall x \in \mathbb{R}$.
Show that $f$ is a constant function.
\end{problem}
\begin{solution}
	For all real $x$, by the condition we have $f(x)=f(3\cdot(x/3))=f(x/3)$ and continuing for $n$ steps we have $f(x)=f(x/3^n)$. Now from the continuity of $f$,
	$$f(x)=\lim_{n\to\infty}f(x/3^n)=f\left(\lim_{n\to\infty}x/3^n\right)=f(0)$$
	which makes $f$ a constant function as $\forall x\in\bR,f(x)=f(0).$
\end{solution}
\begin{problem}\label{prob:5-5}
Let $g: \mathbb{R} \to \mathbb{R}$ be a continuous function satisfying
$g(x + y) = g(x) + g(y), \forall x,y \in \mathbb{R}$.
Show that $g = cx$ for some $c \in \mathbb{R}$.
\end{problem}
\begin{solution}
Just as in \hyperref[prob:2-10]{Problem 2.10.} (problem 10 in the 2nd assignment) we can show that for any rational $r$ we have $f(r)=f(1)\cdot r$ and as given any real number $x$ there exists a sequence of rationals $\{x_n\}$ converging to it we see that,
$$f(x)=f\left(\lim_{n\to\infty}x_n\right)=\lim_{n\to\infty}f(x_n)=\lim_{n\to\infty}f(1)\cdot x_n=f(1)\cdot\lim_{n\to\infty}x_n=f(1)\cdot x$$
where we used the continuity of $f$ in the second inequality and a basic fact from the algebra of limits in the fourth. Thus we see that $c=f(1)$ and $f(x)=f(1)\cdot x$ for all $x\in\bR$
\end{solution}
\begin{problem}\label{prob:5-6}
Let $I$ be an interval in $\mathbb{R}$. A function $f : I \to \mathbb{R}$, is said to be convex if
$f(\lambda x + (1 - \lambda)y) \leq \lambda f(x) + (1 - \lambda)f(y), \forall x < y \text{ in } I, 0 < \lambda < 1$.
(i) Show that if $f : (0, 1) \to \mathbb{R}$ is convex then for $0 < s < t < u < 1$,
\[ \frac{f(t) - f(s)}{t-s} \leq \frac{f(u) - f(s)}{u-s} \leq \frac{f(u) - f(t)}{u-t}. \]
(ii) Show that if $f : (0, 1) \to \mathbb{R}$, is convex then it is continuous.
(iii) Show that a convex function $g : [0, 1] \to \mathbb{R}$, need not be continuous.
\end{problem}
\begin{solution}
Rearranging the first inequality we get the following equivalent inequality,
	$$\frac{f(t)}{t-s}-\frac{f(u)}{u-s}\leq\frac{f(s)}{t-s}-\frac{f(s)}{u-s}=\left(\frac{u-t}{(t-s)(u-s)}\right)f(s)$$
	$$\iff f(t)\leq \frac{u-t}{u-s}\cdot f(s)+\frac{t-s}{u-s}\cdot f(u)$$
	Which is true from the definiton of convexity via $\lambda=(u-t)/(u-s)\in(0,1)$.
	The second inequality is proven similarly. For $(ii),$ we will show that for any $x\in(0,1)$, the right and left limits of $f(yi)$ as $y$ tends to $x$ is zero which will imply that the limit exists and is zero as well, proving continuity. Without loss of generality assume that $x<y,$ then choose $u,v,w\in(0,1)$ such that $u<x<y<v$. Now by the previous result we see that, 
	$$\frac{f(x)-f(u)}{x-u}\leq\frac{f(y)-f(x)}{y-x}\leq\frac{f(v)-f(y)}{v-y}\leq\frac{f(w)-f(v)}{w-v}$$
	$$\implies \frac{f(x)-f(u)}{x-u}\cdot(y-x)\leq f(y)-f(x)\leq (y-x)\cdot\frac{f(v)-f(y)}{v-y}\leq(y-x)\cdot\frac{f(w)-f(v)}{w-v}$$
	Now, using the squeeze theorem, as $y$ tends to $x$ from the right we get, 
	$$\lim_{y\to x;y>x}(f(y)-f(x))=0$$
and the same can be said for the left limit using a very similar arguement and we are done. 
	For (iii) we can consider the following function,
	$$f(x):=\begin{cases}1\text{ if }0<x\leq 1\\ 0\text{ if } x=0\end{cases}$$
		To show that this is convex we will consider the inequality as in the definition on all points $x,y\in[0,1].$ The inequality is clearly always true for $0<x,y$ as the function is constant on $(0,1]$ and when some $x,y$ is $0$, wlog say $x=0,$ then, $$f(\lambda x+(1-\lambda)y)=f((1-\lambda)y)\leq \underbrace{\lambda f(x)}_{=0}+(1-\lambda)f(y)$$
		for both $y=0$ and $y>0$ as in the first case we have $0\leq 0$ and in the second, as $1-\lambda>0,1=1.$ 
\end{solution}
\begin{problem}\label{prob:5-7}
Show that there is no continuous function $f : \mathbb{R} \to \mathbb{R}$, such that for every $y \in \mathbb{R}$, there are exactly two real numbers $x_1, x_2$, such that $f(x_1) = f(x_2) = y$.
\end{problem}
\begin{solution}
	For the sake of contradiction assume that there is some $f$ that is continuous and also satisfies said criterion. Let $x<y$ be the reals such that $f(x)=f(y)=0$. 
	Without loss of generality we can asssume $x<y$. Now consider the interval $[x,y]$, as $f$ is continuous it must have some it must achieve its extrema in this interval at some points in the interval. 
	Unless there is an extrema which is achieved only in $(x,y)$ we can easily see that $f$ is constant in this interval which gives us much more than two, infact infinitely many reals where $f$ is zero which is not allowed as per the assumption. Again without loss of generality assume that we reach a maxima at $u\in(x,y).$
	We claim that $u$ is the unique such point in $[x,y].$
	For a proof of this claim, for the sake of contradiction asssume without loss of generality that we also reach a maxima at $v\in(u,y]$.
	Now $f$ can not be continous on the interval $[u,v]$ so there is some $w\in(u,v)$ such that $0<f(w)<f(u)$ where the first inequality can be achieved due to continuity because by construction $f(u)=f(v)>0$ and if we did not have some $w$ it would produce jump discontinuities in $(u,v).$ 
	Now for any $\lambda\in(f(w),f(u))$ we can find a real $z$ with $f(z)=\lambda$ in all three of the intervals : $(x,u),(u,v),(v,y)$ by IVT, which contradicts the assumption as there should be exactly two such $z$ and the claim is proved. By the condition on $f$, there must be another $g\notin[x,y]$ such that $f(u)=f(g)$. Without loss of generality assume that $g>y,$ then for any $\lambda\in(0,f(u))$ we can find some $z$ with $f(z)=\lambda$ from all of these three intervals : $(x,u),(u,y),(y,g)$ by IVT which is a contradiction and we are done.
\end{solution}
\begin{problem}\label{prob:5-8}
Fix $n \in \mathbb{N}$. Let $x_1, x_2, \dots, x_n$ be $n$ distinct real numbers and let $y_1, y_2, \dots, y_n$ be another n-tuple of not-necessarily distinct real numbers. Show that there is a unique polynomial $p$ of degree $(n - 1)$ such that $p(x_k) = y_k$ for $1 \leq k \leq n$.
Hint: Consider
\[ p(x) = \sum_{j=1}^{n} y_j \frac{\prod_{i\neq j} (x - x_i)}{\prod_{i\neq j} (x_j - x_i)}. \]
\end{problem}
\begin{solution}
	The polynomial $p$ provided in the hint is clearly of degree atmost $n-1$ and when evaluated at $x_i$, all but the $i$'th term in the sum vanish and in this term the numerator and denominator of the fraction cancel out leaving only $y_i$ and thus $p(x_i)=y_i.$ Assume $p,q$ both have degree atmost $n-1$ and they both are polynomials which satisfy the condition in the problem, then we see that $p-q$ has $n$ zeroes. $p-q$ clearly has degree atmost $n-1$ and thus for it to have $n>n-1$ zeroes, by the fundamental theorem of algebra we see that $p-q$ must be identically zero i.e. $p=q$ and hence such a polynomial is unique.
\end{solution}
\begin{problem}\label{prob:5-9}
Let $a < b$ be real numbers. Let $f: (a, b) \to \mathbb{R}$ be a twice differentiable function. Show that for any $x_0 \in (a, b)$
\[ f''(x_0) = \lim_{h\to 0} \frac{f(x_0 + h) - 2f(x_0) + f(x_0 - h)}{h^2}. \]
\end{problem}
\begin{solution}
$f$ being twice differentiable makes it continuous and thus the numerator tends to $0$ as $h\to 0$ and the denominator also tends to zero as $h\to 0$ and also the derivative of the denominator is nonzero near $0$,except at $0.$ 
Let the numerator be $X(h)$ and denominator $Y(h)$, then, 
$$\frac{X'(h)}{Y'(h)}=\frac{f'(x_0+h)-f'(x_0-h)}{2h}$$
$$=\frac{f'(x_0+h)-f'(x_0)}{2h}-\frac{f'(x_0)-f(x_0-h)}{2h}\xrightarrow{h\to 0}\frac{f''(x_0)}{2}+\frac{f''(x_0)}{2}=f''(x_0)$$
and we can thus conclude via LHopital's rule. 
\end{solution}
\begin{problem}\label{prob:5-10}
(i) For $x \in \mathbb{R}$, show that $\sum_{n=0}^{\infty} \frac{x^n}{n!}$ converges absolutely. (ii) Define $e : \mathbb{R} \to \mathbb{R}$ by $e(x) = \sum_{n=0}^{\infty} \frac{x^n}{n!}$. Show that $e(x + y) = e(x)e(y), \forall x, y \in \mathbb{R}$.
\end{problem}
\begin{solution}
	We can say, using \hyperref[prob:2-9]{Problem 2.9.} that for a sequence of positive numbers $x_n$ that converge to $x$, that $\sqrt[n]{x_1x_2\ldots x_n}\xrightarrow{n\to\infty}x$ using that result on $\log(x_n)$ and using the continuity of $\log$ on its domain. We know from \hyperref[prob:2-5]{Problem 2.5.} that the limit $(1+1/n)^{n}$ as $n$ tends to infinity exists and using the binomial theorem we clearly see that $(1+1/n)^n\geq 1+n(1/n)=2$ so the limit is positive, call this $\tt{not_e}.$ Now consider the product,
	$$\prod_{k=1}^n\left(1+\frac{1}{k}\right)^k=\frac{2^1}{1^1}\cdot\frac{3^2}{2^2}\cdots\frac{(n+1)^n}{n^n}=\frac{(n+1)^n}{n!}$$
Using the results mentioned above we have,
	$$\frac{n+1}{(n!)^{1/n}}=\sqrt[n]{\prod_{k=1}^n\left(1+\frac{1}{k}\right)^k}\xrightarrow{n\to\infty}\tt{not_e}$$
	For (i), we will show that for any $x\in\bR$ this series is absolutely convergent using the root test,
	$$\lim_{n\to\infty}\sqrt[n]{|x^n/n!|}=\lim_{n\to\infty}\frac{|x|}{(n!)^{1/n}}=\lim_{n\to\infty}\frac{|x|}{n+1}\cdot\frac{n+1}{(n!)^{1/n}}$$
	$$=\left(\lim_{n\to\infty}\frac{|x|}{n+1}\right)\cdot\lim_{n\to\infty}\frac{n+1}{(n!)^{1/n}}=0\cdot\tt{not_e}=0<1$$
	so we are done. For (ii), using the cauchy product theorem for product of sums, atleast one of which is absolutely convegrent we have,
	$$e(x)e(y)=\sum_{n=0}^{\infty}\sum_{k=0}^n\frac{x^k}{k!}\frac{y^{n-k}}{(n-k)!}=\sum_{n=0}^\infty\frac{1}{n!}\sum_{k=0}^n\binom{n}{k}x^ky^{n-k}=\sum_{n=0}^{\infty}\frac{(x+y)^n}{n!}=e(x+y)$$

\end{solution}
\begin{problem}\label{prob:5-11}
Let $h : \mathbb{R} \to \mathbb{R}$ be the function $h(x) = \sin(x)$ (Here we are assuming your familiarity with trigonometric functions). Show that the remainder term in Taylor's theorem converges to 0 as $n \to \infty$ for every $x_0$ and $x$.
\end{problem}
\begin{solution}
	If we keep taking the derivative if $h$ we can easily see that $|h^{(n)}(x)|$ is either $|\sin x|$ or $|\cos x|$, eitherways its bounded by $1$. Now given any $x,x_0$ the $n-th$ remainder term, say $R_n$ tends to zero as follow,
	$$|R_n|\leq\frac{|x_0-x|^n}{n!}\cdot\sup_{x\in\bR}|h^{(n)}(x)|=\frac{|x-x_0|^n}{n!}\xrightarrow{n\to\infty}0$$
	just as we did in \hyperref[prob:5-10]{Problem 5.10.}.
\end{solution}
\begin{problem}\label{prob:5-12}
Consider the function $h : [0, \infty) \to \mathbb{R}$ defined by $h(x) = \frac{x}{1+x^2}$. Determine the set $\{h(x) : x \in [0, \infty)\}$.
\end{problem}
\begin{solution}
	Clearly $h(x)\geq 0$ with equality at $0$ and using AM-GM, $h(x)\leq 1/2$ with equality at $x=1$. By IVT we can say that it also achieves every value in between and hence the set is $[0,1/2]$$
\end{solution}
\begin{problem}\label{prob:5-13}
Let $v : \mathbb{R} \to \mathbb{R}$ be the function $v(x) = x^3 - 6x^2 + 9x$, for all $x$. Determine the set: $\{x : v(x) > 0\}$.
\end{problem}
\begin{solution}
	We can factor this into $x(x-3)^2$, so it is zero at $0$ and $2$. For $x\neq 0,2$ it has the same sign as $x$ as $(x-3)^2$ is positive. Thus the set is $(0,\infty)\setminus\{2\}.$
\end{solution}
\begin{problem}\label{prob:5-14}
Let $f: [0,1] \to [0, 1]$ be a continuous strictly increasing bijection. Assume that $x < f(x)$ for all $x \in (0,1)$. Fix $x_0 \in (0,1)$. Define $x_n$ for $n \in \mathbb{Z}$ by $x_n = f^n(x_0)$. Show that:
(i) $0 < x_m < x_n < 1$ for $m < n$ in $\mathbb{Z}$;
(ii) $\lim_{n\to\infty} x_n = 1$ and $\lim_{n\to-\infty} x_n = 0$;
(iii) For every $n \in \mathbb{Z}$, $f$ maps $[x_n, x_{n+1}]$ bijectively to $[x_{n+1}, x_{n+2}]$.
\end{problem}
\begin{solution}
	As $f$ is strictly increasing we see that we must have $f(0)=0$ and $f(1)=1$ and as its also a bijection, $0$ or $1$ never appear in $f$ orbits i.e. $\{f^n(x):n\in\bZ\}$ of any points in $(0,1)$. Thus we can see that for any $m<n$, 
	$$f^n(x_0)=\underbrace{f(f^{n-1}(x_0)>f^{n-1}(x_0)>\ldots>f^m(x_0)}_{n-m\text{ inequalities}}$$
	and as $1$ and $0$ never appear in the orbit of $x_0$ we see that $0<f^n(x_0)<1$ for all $n\in\bZ$ and together it reads $0<x_m<x_n<1$, we have thus proved $(i)$. For (ii), as $n\to\infty,\{x_n\}$ must converge to a limit as its increasing and bounded above, say it converges to $x$, then $0<x\leq 1$. Using the continuity of $f$ we have can show that this is a fixed point as follows 
	$$f(x)=f\left(\lim_{n\to\infty}x_n\right)=\lim_{n\to\infty}f(x_n)=\lim_{n\to\infty}x_{n+1}=\lim_{n\to\infty}x_n=x$$
	so it has to be $1$. We can similarly show that $x_n\xrightarrow{n\to-\infty}x_n=0$ as well. For every $n\in\bZ$ we see that $x_n<x_{n+1}$ and thus $x_{n+1}=f(x_n)<f(x_{n+1})=x_{n+2}$ so we have $f([x_n,x_{n+1}])\subseteq[x_{n+1},x_{n+2}]$ using the fact that $f$ is increasing. Using IVT we can say that we have equality here i.e. $f([x_n,x_{n+1}])=[x_{n+1},x_{n+2}]$ and the $f$ is clearly still a bijection here, this proves (iii). 
\end{solution}
\begin{problem}\label{prob:5-15}
	Consider the set up of the previous question. Fix $y_0 \in (x_0, x_1)$. Set $y_n = f^n(y_0)$ for $n \in \mathbb{Z}$. Let $h : [x_0, y_0] \to [y_0, x_1]$ be a continuous strictly increasing bijection. Define $g : [0, 1] \to [0,1]$ by $g(0) = 0, g(1) = 1$,
\[ g(t) = h(t) \quad \forall t \in [x_0, y_0]; \]
\[ g(t) = f \circ h^{-1}(t) \quad \forall t \in [y_0, x_1]; \]
and more generally, for $n \in \mathbb{Z}$, define
\[ g(t) = f^n \circ h \circ f^{-n}(t) \quad \forall t \in [x_n, y_n] \]
and
\[ g(t) = f^{n+1} \circ h^{-1} \circ f^{-n}(t) \quad \forall t \in [y_n, x_{n+1}]. \]
Then show that:
(i) For every $n$, $g$ maps $[x_n, y_n]$ bijectively to $[y_n, x_{n+1}]$ and it maps $[y_n, x_{n+1}]$ bijectively to $[x_{n+1}, y_{n+1}]$.
(ii) $g$ is a strictly increasing continuous bijection.
(iii) $f = g \circ g$.
This shows that $f$ has infinitely many square roots.
\end{problem}
\begin{solution}
	We know that inverses of continuous bijections in $\bR$ are also continuous bijections and both of these are strictly increasing.
	From the previous problem we can see that $f^{-n}$ bijectively maps $[x_n,y_n]$ to $[x_0,y_0]$, after this $h$ maps $[x_0,y_0]$ to $[y_0,x_1]$ bijectively and at the end $f^n$ maps this bijectively to $[y_n,x_{n+1}]$, thus $g$ maps $[x_n,y_n]$ bijectively to $[y_n,x_{n+1}]$. Using similar arguements we can show that $g$ also bijectively maps $[y_n,x_{n+1}]$ to $[x_{n+1},y_{n+1}]$ and (i) is proven.
	We can see that 
	$$\mathfrak J:=\{[x_n,y_n),[y_n,x_{n+1}):n\in\bZ\}\cup\{\{0\},\{1\}\}$$
	is a partition of $[0,1]$, this is because $\{[x_n,x_{n+1}):n\in\bZ\}\cup\{\{0\},\{1\}\}$ is a partition (this can be deduced from parts (i) and (ii) of \hyperref[prob:5-14]{Problem 5.14.}) and we just partition each of the intervals into two intervals. 
	As $g(0)=0,g(1)=1$ and, 
	$$g([x_{n},y_{n}))=[x_{x+1},y_{n}]\setminus\{g^{-1}(y_n)\}=[y_{n},x_{n+1})$$
	As, $g(y_n){=f^n(h(f^{-n}(y_n)))=f^n(h(y_0))=f^n(x_1)=x_{n+1}}$. 
	Similarly, for the other type of intervals, as, $g(x_{n+1})=f^{n+1}(h^{-1}(f^{-n}(x_{n+1}))=f^{n+1}(h^{-1}(x_1))=f^{n+1}(y_0)=y_{n+1}$ we get,
	$$g([y_n,x_{n+1}))=[x_{n+1},y_{n+1})$$
	Now, looking at $g$ as a set function defined by $g(S):=\{g(x):x\in S\}$ we can clearly see that $g$ is a bijection on $\mathfrak I$ and as its a bijection on each set in $\mathfrak I$ which is a partition of $[0,1]$, its clearly a bijection on $[0,1]$ as a whole. Thus we have shown that $g$ is continous and a bijection, thus it must be strictly increasing as well hence proving (ii). For (iii), firstly $(g\circ g)(0)=0=f(0)$ and $(g\circ g)(1)=1=f(1)$, now if some $x\in(0,1)$ is in an interval of form $[x_n,y_n]$ then when $g$ is first applied to it, it acts as $f^n\circ h\circ f^{-n}$ and sends it to the interval $[y_n,x_{n+1}]$ as seen in (ii) and thus on the second application $g$ acts as $f^{n+1}\circ h^{-1}\circ f^{-n}$ as in the definition of $g$, thus,
	$$(g\circ g)(x)=(f^{n+1}\circ h^{-1}\circ f^{-n}\circ f^n\circ h\circ f^{-n})(x)=(f^{n+1}\circ h^{-1}\circ(f^{-n}\circ f^n)\circ h\circ f^{-n})(x)$$
	$$=(f^{n+1}\circ h^{-1}\circ\operatorname{Id}\circ h\circ f^{-n})(x)=(f^{n+1}\circ h^{-1}\circ(\operatorname{Id}\circ h)\circ f^{-n})(x)=(f^{n+1}\circ h^{-1}\circ h\circ f^{-n})(x)$$
	$$=(f^{n+1}\circ(h^{-1}\circ h)\circ f^{-n})(x)=(f^{n+1}\circ\operatorname{Id}\circ f^{-n})(x)=(f^{n+1}\circ(\operatorname{Id}\circ f^{-n}))(x)=(f^{n+1}\circ f^{-n})(x)$$
	$$=f(x)$$
	and when $x$ is in an interval of form $[y_n,x_{n+1}]$ using similar arguements we have that,
	$$(g\circ g)(x)=(\underbrace{f^{n+1}\circ h\circ f^{-(n+1)}}_{\text{as }g(x)\in[x_{n+1},y_{n+1}]}\circ f^{n+1}\circ h^{-1}\circ f^{-n})(x)=(f^{n+1}\circ f^{-n})(x)=f(x)$$
	Thus we have shown that $g\circ g=f$ on $[0,1]$ and we are done.
\end{solution}
\end{document}
